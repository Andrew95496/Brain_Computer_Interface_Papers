\documentclass[english]{def}

\titulo{ACTION POTENTIALS TO BITS}
          

\title{Neuroscience and Electrodynamics}


\author{Andrew Leacock
}


\begin{document}

\maketitle



\englishtitle

\begin{abstract}
	This series of paper will document my growth in the field of Brain Computer Interfaces (BCI). This paper will document my research from April 7\textsuperscript{th}, 2022 - April 14\textsuperscript{th}, 2022. It will cover neurons, electrical signals, and Coulomb’s Law.
\end{abstract}

\begin{keywords}
	Action potential, Axons, Cerebral Cortex, Glial cells, Goldman equation, Neurons, Vector Analysis
\end{keywords}

\let\thefootnote\relax\footnotetext{\hspace*{-5mm}}

%~~~~~~~~~~~~~~~~~~~~~~~~~~~~~~~~~~~~~~~~
% Seções
%~~~~~~~~~~~~~~~~~~~~~~~~~~~~~~~~~~~~~~~~

% Introdução
\section{NEURONS}
The cells in the nervous system can be divided into two groups nerve cells \textbf{(neurons)} and supporting cells. Neurons are specialized cells for \emph{transporting electrical signals} throughout the body. Supporting cells are not capable of transporting electrical signals. In terms of structure neurons are pretty similar to other cells containing a nucleus, mitochondria, ribosomes, etc. What differentiates neurons from other cells in the body are the presence of \textbf{dendrites} and an \textbf{axon}. Dendrites are \emph{branch like structures that arise from the cell body}. The Axon is \emph{where \textbf{action potentials} are generated and where input from the dendrites are read out}. The geometry of neurons range across a spectrum from a cell body with an axon, to a neurons with hundreds of dendrites. The number of inputs a single neuron can receive range from 1 to about 100,000. The axonal mechanism that carries signals across the body are what action potentials are. Information encoded in the action potential are passed on by the \textbf{synaptic transmission}. The information gets sent to the target cells such as other neurons in the brain, spinal cord, muscles, etc. 

\subsection{NEUROGLIAL CELLS}

\footnotetext{Myelin affect the speed of action potential conduction}

\textbf{Glial cells} do not carry electrical signals but some of there supportive function help maintain the electrical signaling of neurons.
Glia are smaller than neurons and do not have dendrites or axons.\\\\
3 types of glial cells:

\subsubsection{\textbf{Astrocytes}} located in the brain and spinal cord, maintain the chemical environment neuronal signaling.
\subsubsection{\textbf{Oligodendrocyte}} located in the central nervous system, lay a laminated wrapping called \textbf{myelin} around some of the axons.
\subsubsection{\textbf{Microglial}}  located in the central nervous system, help repair nerve damage.

%Body Text
\section{ELECTRICAL SIGNALS}

Naturally neurons are not good conductors of electrical signals. Neurons have evolved complex mechanisms for generating electrical signals based on the flow of ions across their plasma membranes. Usually a neuron generates a negative potential called the \textbf{resting membrane potential}. The action potential is what makes the trans-membrane temporarily positive. Action potentials propagate along the length of the axon and is the fundamental  electrical signal of the axon. The resting potential and action potential show the selective permeability of neurons.

\subsection{ELECTRICAL POTENTIALS}
Neurons are not good electrical conductors in order to make up for that deficiency neurons use action potentials to boost the electrical spike of the neuron. \textbf{Micro-electrodes} are used to record the electrical potential across the plasma membrane. Neurons generate a constant negative voltage when at rest which is called the resting membrane potential. Depending on the neuron being examined the voltage will typically be from -40 to 90 mV.
To elicit an action potential in the laboratory you would insert a second micro-electrode and connect it to a battery. If the charge sent through the battery causes the membrane to be more than the \textbf{threshold potential} an action potential will occur. Action potentials last for about 1 ms. Larger currents do not elicit larger action potentials. If the amplitude or duration of a current is increased sufficiently, multiple action potentials occur. 

\subsection{IONIC MOVEMENTS}
Electrical potentials occur across neuronal membranes because of the the different concentration of specific ions across the neuronal membrane and because the membranes are selectively permeable. Those two reason are because of two kinds of protein  in the cell membrane. The two proteins are \textbf{ion pumps} and \textbf{ion channels}. With the ion pumps and ion channels working together it creates cellular electricity. Ions move down their concentration gradient and electrical gradient. The greater the concentration the stronger the push. Ions will flow to the stronger gradient. The equilibrium potential of an ion determines its influence on the membrane potential.

\subsection{FORCES}
The electrical potential generated across the membrane during \textbf{Electrochemical equilibrium} also called \textbf{equilibrium potential}, can be predicted with the \textbf{Nernst equation}.

\[E_x= \frac{R T}{z F} \ln \frac{[X]_{out}}{[X]_{in}}\]

\noindent\textbf{E} = equilibrium potential for any ion X\\
\textbf{R} = the universal gas constant (8.314 J/K mol)\\
\textbf{T} = absolute temperature in Kelvin\\
\textbf{z} = valence of the permeant ion\\
\textbf{F} = Faraday constant the amount of electrical charge in a mole of a univalent ion (96485.3321 C / mol)\\\\

There is another equation for a more complex ion environment such as when the different types of ion are not distributed equally. In this case it depends on relative permeability. If the membrane is more permeable to $K^+$\footnote{$K^+$ = Potassium} the potential approaches -58 mV. If the membrane is more permeable to $Na^+$\footnote{$Na^+$ = Sodium} the potential approaches +58 mV. For this we use the \textbf{Goldman equation}.

\[ V={\frac {RT}{F}}\ln {\left({\frac {\sum _{i}^{n}P_{M_{i}^{+}}[M_{i}^{+}]_{\mathrm {out} }+\sum _{j}^{m}P_{A_{j}^{-}}[A_{j}^{-}]_{\mathrm {in} }}{\sum _{i}^{n}P_{M_{i}^{+}}[M_{i}^{+}]_{\mathrm {in} }+\sum _{j}^{m}P_{A_{j}^{-}}[A_{j}^{-}]_{\mathrm {out} }}}\right)}\]

\noindent\textbf{V} = voltage across the membrane\\
\textbf{P} = permeability of the membrane to each ion of interest\\

% \section{VECTOR ANALYSIS}

% A \textbf{vector} has a direction and a magnitude(length) such as velocity, acceleration, force, momentum, etc. A \textbf{scalar} has a magnitude but no direction such as mass, charge, density, temperature, etc. The are two ways to multiply vectors cross Product and dot Product. The products of dot products are scalars and the products of cross product are vectors.

% \captionof*{figure}{\centering{Dot Product}}
% \[\hat{a} + \hat{b} = c\]

% \captionof*{figure}{\centering{Cross Product}}
% \[\hat{a} + \hat{b} = \hat{c}\]

% In electrodynamics you will frequently encounter problems involving a \textbf{source point r'}, where an electric charge is located \emph{q}, and a \textbf{field point r}, at which you are calculating the electric field \emph{Q}. The vector that is the vector from the source point to the field point is called the \textbf{separation vector} (\textbf{sv}).

% \captionof*{figure}{\centering{Standard}}
% \[\textbf{sv} = r - r'\]

% \captionof*{figure}{\centering{Magnitude}}
% \[sv = |r - r'|\]

% If \emph{q} and \emph{Q} have the same sign the force is repulsive, and attractive if the signs are the opposite. 

% \section{ELECTROSTATICS}

% The fundamental problem of electrodynamics is to solve what some forces do some charges exert on another. The solution to this problem is facilitated by the \textbf{principle of superposition}, which states that the interaction between any two charges is completely unaffected by the presence of others. We will start with \textbf{electrostatics} where all source charges are stationary.
% \[ \textbf{F}_{Q} = \sum \textbf{F}_{QI}\]
% \footnote{F = Force}

%  \subsection{COULOMB'S LAW}
 
%  Coulomb's law answers, what is the force on a test charge \emph{Q} due to a single point charge \emph{q}, that is at rest a distance \textbf{sv} away.
 
 
 
%  \captionof*{figure}{\centering{Into to Electrodynamics Book}}
% \[\textbf{F}=\frac{1}{4\pi\epsilon_{0}}\frac{\emph{q}\emph{Q}}{sv^2}\textbf{\textbf{$\hat{sv}$}}\]
 
%  \captionof*{figure}{\centering{Wikipedia}}
%  \[\textbf{F}=k_{e}{\frac {q_{1}q_{2}}{r^{2}}}\]
 
 




% ---------------------------------------------------------------------------------------------------------
\section*{GLOSSARY}
\noindent\textbf{Ion Channels} allows certain kind of ions move across the membrane in the direction of their concentration gradients\\\\
\textbf{Ion Pumps} moves ions in and out of cells based on their concentration gradients\\\\


\begin{thebibliography}{9}

\bibitem{}Griffiths, D. J. (2018). \emph{Introduction to electrodynamics}. Cambridge University Press.

\bibitem{}
Kandel, E. R., Schwartz, J. H., \& Jessell, T. M. (1991). \emph{Principles of Neural Science. Elsevier}. Columbia University 

\bibitem{}
Klabunde, E, Richard PhD. (n.d.). (2022) \emph{Action potentials. Image for Cardiovascular Physiology Concepts} https://www.cvphysiology.com/Arrhythmias

\bibitem{}
Pete Meighan. (2019). \emph{The Nernst Equation and Equilibrium Potentials in Physiology}. YouTube. https://youtu.be/ZVUWyDzmApg

\bibitem{}
Physics, Real. (2015). \emph{Introduction to Electrodynamics}. YouTube. https://youtube.com/playlist?list=PLDDEED00333C1C30E 

\bibitem{}
Purves, D., Augustine, G. J., Fitzpatrick, D., Hall, W. C., LaMantia, A.-S., \& White, L. E. (2012) \emph{Neuroscience}, Sinauer Associates, Publishers.

 






\end{thebibliography}




\end{document}
